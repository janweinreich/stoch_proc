\documentclass[12pt]{article}
\usepackage[utf8]{inputenc}
\usepackage{geometry}
\usepackage{svg}
\usepackage{float}
\usepackage{caption}
\usepackage{amsmath,amsthm,amsfonts,amssymb,amscd}
\usepackage{fancyhdr}
\usepackage{titlesec}

\pagestyle{empty}
\titleformat*{\section}{\large\bfseries}

%
\geometry{
 a4paper,
 total={170mm,240mm},
 left=20mm,
 top=30mm,
 }

\date{}
%Bitte ausfüllen
\newcommand\course{Stochastic Processes in Physics}
\newcommand\hwnumber{1}
\newcommand\Name{Jan Weinreich}
\newcommand\Neptun{Uni Vienna}

%Matheinheiten
\newcommand\m{\:\textrm{m}}
\newcommand\M{\:\Big[\textrm{m}\Big]}
\newcommand\mm{\:\textrm{mm}}
\newcommand\MM{\:\Big[\textrm{mm}\Big]}
\newcommand\un{\underline}
\newcommand\s{\:\textrm{s}}
\newcommand\bS{\:\Big[\textrm{S}\Big]}
\newcommand\ms{\:\frac{\textrm{m}}{\textrm{s}}}
\newcommand\MS{\:\Big[\frac{\textrm{m}}{\textrm{s}}\Big]}
\newcommand\mss{\:\frac{\textrm{m}}{\textrm{s}^2}}
\newcommand\MSS{\:\Big[\frac{\textrm{m}}{\textrm{s}^2}\Big]}

%Bitte nicht einstellen
\renewcommand{\figurename}{Abbildung}
\renewcommand{\tablename}{Tabelle}
\pagestyle{fancyplain}
\headheight 35pt
\lhead{\Name\\\Neptun}
\chead{\textbf{\Large Hausaufgabe \hwnumber}}
\rhead{\course \\ \today}
\lfoot{}
\cfoot{}
\rfoot{\small\thepage}
\headsep 1.5em

\begin{document}



%include svg files


\section{Random Walks}
\begin{figure}[H]
  \centering
    \includegraphics[width=0.45\linewidth]{./figures/JanWeinreich_P1_1a.png}
    \includegraphics[width=0.45\linewidth]{./figures/JanWeinreich_P1_1b.png}
    \caption{Left: Mean squared deviation of a random walker with gaussian noise, 
             numerical (num) and theoretical curve (ex).  
             Right: Histogram of random walkter position $x$ after $n=2, 5, 20, 100$ steps and solution to the diffision 
             Eq.~(\ref{eq:diff1}) for $t= n=100$.}
    \label{fig:abb1}
  \end{figure}


The solution for the probability density $\rho$ for a particle
initially at $x=0$ for $t=0$ to be located at position $x$ at time $t$ is given by

\begin{align}
    \rho (x, t) = \frac{1}{\sqrt{4 \pi t}} \exp{   \left( -\frac{x^2}{4Dt} \right) }
\end{align}  
The mean squared deviation $  \langle x^2 \rangle $ is the second momentum of the 
probability distribution since the first momentum is zero because $\mu(\rho)$ = 0.
Thus we have,
\begin{align}
  \langle x^2 \rangle = \int_{-\infty}^{+\infty} dx \rho (x, t) x^2 = 2 Dt
  \label{eq:diff1}
\end{align}
In Fig.~\ref{fig:abb1} we show Eq.~\ref{{eq:diff1} for $D=1$ and $t= n=100$.





We consider a walker with steps of fixed size $\Delta x = \pm 1 $ with equal probabilities for  both directions.
We introduce a random variable $s_i$:
\[   
s_i = 
     \begin{cases}
       +1 & \text{, if }\Delta x = +1\\
       -1 & \text{, if }\Delta x = -1 \\
     \end{cases}
\]



\begin{align}
  \langle x^2 \rangle  = \sum_{i, j}^n s_i s_j = \sum_{i}^n s_i s_i  + \sum_{i \neq j}^n s_i s_j
  = \sum_{i}^n s_i s_i  = \sum_{i}^n s_{i}^2 = \sum_{i}^n s_i s_i  = \sum_{i}^n 1 = n  
\end{align}
Note that the sum $\sum_{i \neq j}^n s_i s_j$ averages to zero, since the contributions are alternating with
equal positive and negative amplitude.




\begin{figure}[H]
  \centering
    \includegraphics[width=0.45\linewidth]{./figures/JanWeinreich_P1_2b.png}
    \caption{Histogram of the probability density of being at position $x$ after $n=10, 21$ random steps
            Also shown is the Stirling approximation solution to the problem and the corresponding 
            Gaussian case.}
    \label{fig:abb2}
  \end{figure}






In Fig.~\ref{fig:abb2} we see that the main difference between the solution to the diffusion
problem with $\Delta x = \pm 1$ and gaussian noise is the prefactor in the diffision constant.
The mean squared displacement (MSD) from the diffusion equation is twice as large which leads to
a wider spread of $\rho$ compared to the Stirling approximation.

\section{Weird and Strange Walkers}

\begin{figure}[H]
  \centering
    \includegraphics[width=0.45\linewidth]{./figures/trajectory_weirdwalker.png}
    \caption{Trajectory of the 2-dimensional continous weird walker in blue. We find that the 
            motion is somewhat restricted to the area encirceled in red.}
    \label{fig:abb3}
  \end{figure}
In Fig.~\ref{fig:abb3} we show the motion of the 2-dimensional continous weird walker.
We find that the motion appears to be restricted to a certain radius from the origin at $(0,0)$.
To confirm this suspicion we compute the MSD as shown in Fig.~\ref{fig:abb4}. We find that the 
MSD appears to have an upper limit at $\lim_{t \rightarrow \infty} \langle x^2 \rangle (t) = 2$.
Occasionally this barrier can be overcome, but on average the motion is restricted to within a
radius of $\sqrt{2}$ which is shown as a red circle in Fig.~\ref{fig:abb3}. Thus this data cannot
originate from the Brownian motion of a solute particle in water. An alternative explanation would be that
there is an isotropic force that pulls the walkter back to the center.


\begin{figure}[H]
  \centering
    \includegraphics[width=0.45\linewidth]{./figures/msd_weirdwalker.png}
    \caption{MSD of the 2-dimensional continous weird walker.}
    \label{fig:abb4}
  \end{figure}


Next we consider the Strange walker in 2-dim and study a plot of its MSD in Fig.~\ref{fig:abb4}. We find
that the MSD increases linearly as it would be expected for a typical diffusion process (apart from the fact
that the motion is not continous). To compute the MSD the average was computed over all 350 trajectories.

\begin{figure}[H]
  \centering
    \includegraphics[width=0.45\linewidth]{./figures/msd_strangewalker.png}
    \label{fig:abb5}
  \end{figure}


\section{Cauchy Walkers}

We show that the standart-Cauchy distribution is normalized. Start with:
\begin{align}
  \int_{-\infty}^{+\infty} \frac{1}{1+x^2}
\end{align}
Very obvious and trivial substitution $x:=\tan(z)$, thus:
\begin{align}
  \frac{dx}{dz} = \frac{1}{\cos^2(z)} \rightarrow dx = \frac{dz}{\cos^2(z)}~.
\end{align}
This is very helpful as it simplifies the integration:
\begin{align}
  \int_{-\infty}^{+\infty} \frac{1}{1+x^2} =
   \int_{-\infty}^{+\infty} \frac{1}{1+\tan^2(z)}  \frac{dz}{\cos^2(z)} = \int dz = z
\end{align}
Here we use the widely known fact that:
\begin{align}
  1+\tan^2(z) = \frac{1}{\cos^2(z)}
\end{align}
Thus we find that:
\begin{align}
  \int_{-\infty}^{+\infty} \frac{1}{1+x^2} = \arctan(+\infty) - \arctan(-\infty) = \frac{\pi}{2} - (-\frac{\pi}{2} ) = \pi ~.
\end{align}
And thus:
\begin{align}
  \begin{align}
    \frac{1}{\pi} \int_{-\infty}^{+\infty} \frac{1}{1+x^2} = 1
  \end{align}
\end{align}

Next we try the impossible: To compute the average value of the standart Cauchy distribution:
\begin{align}
  \langle x \rangle = \frac{1}{\pi}  \int_{-\infty}^{+\infty} \frac{x}{1+x^2}
\end{align}
The following substitution is used:
\begin{align}
  1+x^2 = u \rightarrow \langle x \rangle = \frac{1}{2 \pi} \int_{-\infty}^{+\infty}  \frac{1}{u} du \\
  =   \log(x^2+1) |_{-\infty}^{+\infty} = \log(\infty) - \log(-\infty)
\end{align}
The log of a negative number is not defined in terms of real numbers. We shall provide a more convincing proof
that the limit is not defined:
\begin{align}
  \langle x \rangle = 
  \lim_{}
\end{align}

\end{document}
