\documentclass[12pt]{article}
\usepackage[utf8]{inputenc}
\usepackage{geometry}
\usepackage{svg}
\usepackage{float}
\usepackage{hyperref}
\usepackage{hyperref,graphicx}    
 
\usepackage{caption}
\usepackage{amsmath,amsthm,amsfonts,amssymb,amscd}
\usepackage{fancyhdr}
\usepackage{titlesec}
 
\pagestyle{empty}
\titleformat*{\section}{\large\bfseries}
 
%
\geometry{
 a4paper,
 total={170mm,240mm},
 left=20mm,
 top=30mm,
 }
 
\date{}
%Bitte ausfüllen
\newcommand\course{Stochastic Processes in Physics}
\newcommand\hwnumber{1}
\newcommand\Name{Jan Weinreich}
\newcommand\Neptun{Uni Vienna}
 
%Matheinheiten
\newcommand\m{\:\textrm{m}}
\newcommand\M{\:\Big[\textrm{m}\Big]}
\newcommand\mm{\:\textrm{mm}}
\newcommand\MM{\:\Big[\textrm{mm}\Big]}
\newcommand\un{\underline}
\newcommand\s{\:\textrm{s}}
\newcommand\bS{\:\Big[\textrm{S}\Big]}
\newcommand\ms{\:\frac{\textrm{m}}{\textrm{s}}}
\newcommand\MS{\:\Big[\frac{\textrm{m}}{\textrm{s}}\Big]}
\newcommand\mss{\:\frac{\textrm{m}}{\textrm{s}^2}}
\newcommand\MSS{\:\Big[\frac{\textrm{m}}{\textrm{s}^2}\Big]}
 
%Bitte nicht einstellen
\renewcommand{\figurename}{Abbildung}
\renewcommand{\tablename}{Tabelle}
\pagestyle{fancyplain}
\headheight 35pt
\lhead{\Name\\\Neptun}
\chead{\textbf{\Large Hausaufgabe \hwnumber}}
\rhead{\course \\ \today}
\lfoot{}
\cfoot{}
\rfoot{\small\thepage}
\headsep 1.5em
 
\begin{document}
 
 
 
%include svg files
 
 
\section{Random Walks}
\begin{figure}[H]
  \centering
    \includegraphics[width=0.45\linewidth]{./figures/JanWeinreich_P1_1a.png}
    \includegraphics[width=0.45\linewidth]{./figures/JanWeinreich_P1_1b.png}
    \caption{Left: Mean squared deviation $\langle x^2 \rangle$ of a random walker with gaussian noise, 
             numerical (num) and theoretical curve (ex) with $\langle x^2 \rangle$ = n.  
             Right: Histogram of random walker position $x$ after $n=2, 5, 20, 100$ steps and solution to the diffusion 
             Eq.~(\ref{eq:diff1}) for $t= n=100$.}
    \label{fig:abb1}
  \end{figure}
 
 
The solution for the probability density $\rho$ for a particle
initially at $x=0$ for $t=0$ to be located at position $x$ at time $t$ is given by
 
\begin{align}
    \rho (x, t) = \frac{1}{\sqrt{4 \pi t}} \exp{   \left( -\frac{x^2}{4Dt} \right) }
\end{align}  
The mean squared deviation $  \langle x^2 \rangle $ is the second moment of the 
probability distribution and also the variance of $\rho$
since the first moment is zero because $\mu(\rho)$ = 0 and
$\text{Var}(\rho) =  \langle x^2 \rangle -  \langle x \rangle^2$ .
Thus we have,
\begin{align}
  \langle x^2 \rangle = \int_{-\infty}^{+\infty} dx \rho (x, t) x^2 = 2 Dt
  \label{eq:diff1}
\end{align}
In Fig.~\ref{fig:abb1} (left) we show the MSD from $10000$ independent Gaussian random walks with $n=100$ steps each.
We find the expected linear increase with the number of steps (time) with a diffusion constant of $D=\frac{1}{2}$.
In Fig.~\ref{fig:abb1} (right) we show Eq.~\ref{eq:diff1} for $D=\frac{1}{2}$ and $t= n=100$ and we see
that the prediction by the diffusion equation matches perfectly to an estimated density distribution obtained 
by the simulation.
\\
 
Now we consider a walker with steps of fixed size $\Delta x = \pm 1 $ with equal probabilities for  both directions.
We introduce a random variable $s_i$:
\[   
s_i = 
     \begin{cases}
       +1 & \text{, if }\Delta x = +1\\
       -1 & \text{, if }\Delta x = -1 \\
     \end{cases}
\]
 
 
 
\begin{align}
  \langle x^2 \rangle  = \sum_{i, j}^n s_i s_j = \sum_{i}^n s_i s_i  + \sum_{i \neq j}^n s_i s_j
  = \sum_{i}^n s_i s_i  = \sum_{i}^n s_{i}^2 = \sum_{i}^n s_i s_i  = \sum_{i}^n 1 = n  
\end{align}
Note that the sum $\sum_{i \neq j}^n s_i s_j$ averages to zero, since the contributions are alternating with
equal positive and negative amplitude.
 
 
 
 
\begin{figure}[H]
  \centering
    \includegraphics[width=0.45\linewidth]{./figures/JanWeinreich_P1_2b.png}
    \caption{Histogram of the probability density of being at position $x$ after $n=10, 21$ random steps
            Also shown is the Stirling approximation solution to the problem and the corresponding 
            Gaussian case.}
    \label{fig:abb2}
  \end{figure}
 
 
 
 
 
 
In Fig.~\ref{fig:abb2} we see that the main difference between the solution to the diffusion
problem with $\Delta x = \pm 1$ and Gaussian noise is the prefactor in the diffusion constant.
 
\begin{align}
  \text{Smoluchowski: }
  \langle x^2 \rangle = n \leftrightarrow \text{ Einstein:}
  \langle x^2 \rangle = 2 D t
\end{align}
Thus effectively the MSD deviations of both solutions only differ by a factor of two.
The mean squared displacement (MSD) of the diffusion equation is twice as large which leads to
a wider spread of $\rho$ compared to the Stirling approximation.
 
 
 
\section{Weird and Strange Walkers}
 
\begin{figure}[H]
  \centering
    \includegraphics[width=0.45\linewidth]{./figures/trajectory_weirdwalker.png}
    \caption{Trajectory of the 2-dimensional continuous weird walker in blue. We find that the 
            motion is somewhat restricted to the area encircled in red.}
    \label{fig:abb3}
  \end{figure}
In Fig.~\ref{fig:abb3} we show the motion of the 2-dimensional continuous weird walker.
We find that the motion appears to be restricted to a certain radius from the origin at $(0,0)$.
To confirm this suspicion we compute the MSD as shown in Fig.~\ref{fig:abb4}. We find that the 
MSD appears to have an upper limit at $\lim_{t \rightarrow \infty} \langle x^2 \rangle (t) = 2$.
Occasionally this barrier can be overcome, but on average the motion is restricted to within a
radius of $\sqrt{2}$ which is shown as a red circle in Fig.~\ref{fig:abb3}. An alternative explanation would be that
there is an isotropic force that pulls the walker back to the center.
Thus this data cannot originate from the Brownian motion of a solute particle in water unless maybe the container of the liquid is just as large as the mentioned barrier.
 
\begin{figure}[H]
  \centering
    \includegraphics[width=0.45\linewidth]{./figures/msd_weirdwalker.png}
    \caption{MSD of the 2-dimensional continuous weird walker.}
    \label{fig:abb4}
  \end{figure}
 
 
Next we consider the Strange walker in 2-dim and study a plot of its MSD in Fig.~\ref{fig:abb4}. We find
that the MSD increases linearly as it would be expected for a typical diffusion process (apart from the fact
that the motion is not continuous). To compute the MSD the average was computed over all 350 trajectories.
 
\begin{figure}[H]
  \centering
    \includegraphics[width=0.45\linewidth]{./figures/msd_strangewalker.png}
    \label{fig:abb5}
  \end{figure}
 
 
\section{Cauchy Walkers}
 
We show that the standard-Cauchy distribution is normalized. Start with:
\begin{align}
  \int_{-\infty}^{+\infty} \frac{1}{1+x^2}
\end{align}
Very obvious and trivial substitution $x:=\tan(z)$, thus:
\begin{align}
  \frac{dx}{dz} = \frac{1}{\cos^2(z)} \rightarrow dx = \frac{dz}{\cos^2(z)}~.
\end{align}
This is very helpful as it simplifies the integration:
\begin{align}
  \int_{-\infty}^{+\infty} \frac{1}{1+x^2} =
   \int_{-\infty}^{+\infty} \frac{1}{1+\tan^2(z)}  \frac{dz}{\cos^2(z)} = \int dz = z
\end{align}
Here we use the widely known fact that:
\begin{align}
  1+\tan^2(z) = \frac{1}{\cos^2(z)}
\end{align}
Thus we find that:
\begin{align}
  \int_{-\infty}^{+\infty} \frac{1}{1+x^2} = \arctan(+\infty) - \arctan(-\infty) = \frac{\pi}{2} - (-\frac{\pi}{2} ) = \pi ~.
\end{align}
And thus:
\begin{align}
  \begin{align}
    \frac{1}{\pi} \int_{-\infty}^{+\infty} \frac{1}{1+x^2} = 1
  \end{align}
\end{align}
 
Next we try the impossible: To compute the average value of the standard Cauchy distribution:
\begin{align}
  \langle x \rangle = \frac{1}{\pi}  \int_{-\infty}^{+\infty} \frac{x}{1+x^2}
\end{align}
The following substitution is used:
\begin{align}
  1+x^2 = u \rightarrow \langle x \rangle = \frac{1}{2 \pi} \int_{-\infty}^{+\infty}  \frac{1}{u} du \\
  =   \log(x^2+1) |_{-\infty}^{+\infty} = \log(\infty) - \log(-\infty)
\end{align}
The log of a negative number is not defined in terms of real numbers. We shall provide a more convincing proof
that the limit is not defined:
\begin{align}
  \langle x \rangle = 
  \lim_{T \rightarrow \infty} \int_{-T}^{aT} \frac{x}{\pi} \frac{1}{1+x^2} dx =
  \lim_{T \rightarrow \infty} \frac{1}{2 \pi} \log\left(\frac{1+(aT)^2}{1+T^2} \right) =  \\
  \frac{1}{2 \pi} \lim_{T \rightarrow \infty} 
   \log\left(  \frac{   T^{2} ( T^{-2}+a^2)}{ T^2 ( T^{-2} +1)} \right) =
   \frac{1}{ 2 \pi} \log(a^2)
\end{align}
Thus the average value can take an arbitrary value and is not uniquely defined.
The same is true for $ \langle | x |  \rangle $.
Next we compute the second moment:
\begin{align}
  \langle x^2 \rangle =\frac{1}{\pi} \int_{-\infty}^{+\infty} \frac{x^2}{1+x^2} dx =
  \frac{1}{\pi} \int_{-\infty}^{+\infty} \frac{1+x^2-1}{1+x^2} dx =
  \frac{1}{\pi} \left(\int_{-\infty}^{+\infty} dx - \pi \right) = \infty
\end{align}
 
Below we show 10 Cauchy-random walks with $n=200$ steps (s. Fig.~\ref{fig:abb6}). We find that many
of the trajectories have a large deviation of more than 1000 from zero.
 
However $x=0$ is the intuitive value one would expect from looking only at the distribution plot of the Cauchy distribution.
This is due to the infinite MSD and because the average value can take arbitrary values as shown above.
 
 
 
\begin{figure}[H]
  \centering
    \includegraphics[width=0.45\linewidth]{./figures/JanWeinreich_P1_4a.png}
    \includegraphics[width=0.45\linewidth]{./figures/JanWeinreich_P1_4b.png}
    \caption{Left: 10 Cauchy random walks. Right: The running average of the particle position for
    over $N=200$ independent trajectories.}
    \label{fig:abb6}
  \end{figure}
 
  Next we consider the running average of the particle positions over $N=200$ independent trajectories.
  \begin{align}
    \bar{x}_{n} = \sum_{i=0}^{n-1} x_{i}/n
  \end{align}
  We find that the running average converges very fast for the Gaussian distribution. For the 
  Cauchy distribution However we find very large deviation from the intuitive but wrong 
  expectation value zero. This is again due to the fact that the expectation value is ill defined
  and due to infinite standard deviation. Thus we cannot expected this quantities to be defined and
  this will not be different for large sampling sizes.
 
 
 
  Next we consider the distribution of the mean values of the 
  Cauchy random walks, shown in Fig.~\ref{fig:abb7}.
  It appears the average value is equally distributed across the real numbers.
  This is plausible because we have shown that the first moment of
  the Cauchy distribution can take arbitrary values.
  This is very counter-intuitive since the most likely interval (maximum of Cauchy) is around zero.
  \begin{figure}[H]
    \centering
      \includegraphics[width=0.45\linewidth]{./figures/JanWeinreich_P1_4c.png}
      \caption{In blue: Distribution of the average positions of 200 independent
      Cauchy Walkers. Orange: Cauchy Distribution, the heavy tail prohibits convergence of the first two moments of the distribution.}
      \label{fig:abb7}
    \end{figure}
 
\end{document}